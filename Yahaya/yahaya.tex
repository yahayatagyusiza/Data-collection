\documentclass[12]{article}

\begin{document}

\title{THE CONCEPT ON MAKERERE UNIVERSITY HOSPITAL MANAGEMENT}
\maketitle
\author{GORLYAN ISAAC   14/U/6365/PS    214003014}

\section{Introduction}
The makerere university hospital has many departments that are to work on the patients who seek medication from the various particular departments and has encountered problems due to complaints submitted in by patients. This concept has concentrated on the students and the services they perceived from the hospital and how they evaluate the management and organization of the hospital.

\section{Background of the concept}
The university hospital has been under great scrutiny by the premise users both students and the people of the surrounding community on the management and the control of how services are carried out and offered to the patients of the recent years. There is typical redundancy by the doctors in the way they don’t value patients’ lives  who surface the services of the university hospital face due to the negligence of the workers who take time doing unrelated works to the well-being of the patients who have come for medication.

\section{Problem statement}
The problem this concept I carried out will address is to see how the university hospital has been of value or of disgrace to the users mainly the students who are attached to and live in various halls of residences. This comprehensive data collection concept will focus on the way the medical staff is of great impact or not to the patients.
There is redundancy in the time it takes for the medical staff in various departments to work on the multiple patients hence need to know from the patients who have ever used the hospital if they were worked on in time regardless of which year of study or course.

\section{Objectives}
\subsection{General objectives}
To solve any form of time lag on the patients who have come for medication
\subsection{Specific objectives}
•	To increase on the operating rate of the hospital management.
•	To save time on it takes of students take waiting to be worked on by the hospital departments.
•	Plan of the management of the hospital after the suggestions they get from the clients.

\section{Methodology}
I used my ODK collect kit installed on my android phone to collect data from the selective students both from female students and male students to have an overview on the way they have a view on the management of the hospital. The ODK collect kit collects information electronically in form of a questionnaire.

\subsection{Scope}
The concept is to concentrate on the students of Makerere University both who ate attached and those who are not attached but have the permission to have service to the university hospital.

\subsection{Significance}
This study is important since it helps the management to maintain and concentrate on the performance of the various university hospital departments.
This study will also important because it will enhance monitoring of how many students surface the hospital services in a particular period of time to see how they can stock for the equipment and plan on how to work on the patients.

\end{document}
